\documentclass[12pt]{article}
\usepackage{../../template}
\title{Velocity Profiles}
\author{niceguy}
\begin{document}
\maketitle

\section{Introduction}

Here I'll derive the equations (with steps) for velocity profiles under different conditions. You can sketch them out in the Jupyter Notebook. Unless if specified, assume steady laminar flow, incompressibility, and uniform shear stress along the pipe. Nothing but basic physics and calculus is required.

\section{Some Background}

\subsection{Reynolds Transport Theorem}

Let $z$ be a property independent of mass, and define

$$Z = \int_V \rho z dV$$

For this subsection, we can assume $z$ refers to temperature, though it can refer to specific momentum and mass (where $Z$ becomes momentum and mass respectively). Now let us define $Z$ to be the sum of all $z$ in some volume of the fluid. For convenience, you can imagine a dyed segment of fluid. We also define a control volume $V$ with inlet area $A_{\text{in}}$ and outlet area $A_{\text{out}}$. The control volume is constant with time, but the fluid travels, meaning some dyed fluid exits the control volume, and some undyed fluid enters. Assuming no flow, the change in $Z$ of the fluid is equal to the change in $Z$ in the control volume. If there is fluid outflow, we should \emph{add} the amount of $Z$ that leaves to the change in the control volume. Assuming no change in temperature, for example, but with fluid flowing out, wee need to add the amount of temperature-mass that leaves to the change in control volume to get the desired result (0). Similarly, you subtract inflow of $Z$, as it is included in the change of $Z$ within the control volume, but it is not part of the fluid we are concerned with. Then

$$\frac{dZ}{dt} = \del{}{t} \int_V \rho z dV - \int_{A_{\text{in}}} \rho zvdA_{\text{in}} + \int_{A_{\text{out}}} \rho zvdA_{\text{out}}$$

where $vdA$, or more explicitly $\vec v d\vec A$, is the volume flow rate (volume per time) of fluid flow. Multiplying it by $\rho$ gives the mass flow rate.

% Momentum Correction Factor?
% discussion of forces

\section{Circular Pipe}

This is the classic example. We use $s$ to denote the path traced by the pipe along the centre of the circle. We can then consider the infinitesimal control volume $\pi r^2ds$, which is the cylindrical segment with height $ds$. \\

Since the area of inflow is identical to that of outflow, for mass to be conserved, average velocity in $u_{\text{in}}$ must be equal to average velocity out $u_{\text{out}}$. Then consider the balance of forces along $s$.

\begin{align*}
    \sum F_s &= \rho \int_A \left(u_{\text{in}}^2 - u_{\text{out}}^2\right) dA \\
             &= \rho K_m \int_A \left(\overline{u_{\text{in}}}^2 - \overline{u_{\text{out}}}^2\right) dA \\
             &= 0
\end{align*}

Note that the momentum correction factor $K_m$ is constant. Hence, there is no net force along $s$. Note that net force is composed of pressure, shear stress, and gravity. Letting $\theta$ be the angle between the horizontal plane and $s$,

\begin{align*}
    pr^2\pi &= (p+dp)r^2\pi + 2\pi rds\tau + \rho gr^2\pi ds\sin\theta \\
    rdp + 2\tau ds + \rho g rds\sin\theta &= 0 \\
\end{align*}

Dividing by $ds$ and substituting $\sin\theta = \frac{dz}{ds}$,

\begin{equation}
    \frac{dp}{ds} + \frac{2\tau}{r} + \rho g \frac{dz}{ds} = 0
\end{equation}

\subsection{Fully Developed Flow}

The above equation holds in a developing flow region, where fluid first enters a pipe, and velocity profile changes with $s$. For developed flow, we can simplify the equation. Consider a cylindrical shell control volume with volume $2\pi rdrds$. Similarly, forces along $s$ vanishes. Due to the no-slip condition, we can assume velocity \textit{decreases} with $r$, which gives us the direction of shear forces.

\begin{align*}
    2\pi rpdr + 2\pi r\tau ds &= 2\pi r(p + dp)dr + 2\pi (r\tau + d(r\tau)) ds + 2\pi r\rho g drds \frac{dz}{ds} \\
    2\pi rdpdr + 2\pi d(r\tau)ds + 2\pi r\rho g drds \frac{dz}{ds} &= 0 \\
    r \frac{dp}{ds} + \frac{d}{dr} r\tau + r\rho g \frac{dz}{ds} &= 0
\end{align*}

Again, $p$ and $z$ do not depend on $r$. Integration with respect to $r$,

$$\frac{r^2}{2} \frac{dp}{ds} + r\tau + \frac{r^2}{2} \rho g \frac{dz}{ds} = C$$

Putting $r=0$, the constant $C=0$. Substituting the definition of $\tau$,

$$\frac{r^2}{2} \frac{dp}{ds} - r\mu \frac{du}{dr} + \frac{r^2}{2} \rho g \frac{dz}{ds} = 0$$

Dividing by $r\mu$ and integrating,

$$u = \frac{r^2}{4\mu} \frac{dp}{ds} + \frac{r^2}{4\mu} \rho g \frac{dz}{ds} + C$$

To solve for the constant, simply use the no-slip condition $u(r=R) = 0$

\begin{equation}
    u = \frac{r^2-R^2}{4\mu} \left(\frac{dp}{ds} + \rho g \frac{dz}{ds}\right)
\end{equation}

Observe that for flow to be fully developed, we assume $\frac{dz}{ds}, \frac{dp}{ds}$ are constants. Then the maximum is obviously at $u_{\text{max}} = u(r=0)$, so the shape is given by

$$\frac{u}{u_{\text{max}}} = 1 - \left(\frac{r}{R}\right)^2$$

Hence velocity profile does not depend on slope.

\section{Parallel Planes}

Assume the top plane has a velocity $U>0$, while the bottom plane is stationary. Let $D$ be the spacing between the planes. Velocity then increases with height $y$. We use an infinitesimal element with width $dx$ and height $dy$. Similarly, for mass to be conserved, $u_{\text{in}} = u_{\text{out}}$, implying there is no momentum flux, or no net force. Defining $m$ to be the (constant) slope,

\begin{align*}
    pdy + (\tau + d\tau)dx &= (p+dp)dy + \tau dx + \rho gm dxdy \\
    d\tau dx &= dpdy + \rho gm dxdy \\
    \frac{d\tau}{dy} &= \frac{dp}{dx} + \rho gm \\
    \mu \frac{d^2u}{dy^2} &= \frac{dp}{dx} + \rho gm \\
    u &= \frac{y^2}{2}\left(\frac{dp}{dx} + \rho gm\right) + Ay + B
\end{align*}

Where again $\frac{dp}{dx}$ is independent of $y$ for the same reasons. Substituting the initial conditions $u(0) = 0, u(D) = U$, we obtain

\begin{equation}
    u = \frac{1}{2\mu} \left(\frac{dp}{dx} + \rho gm\right) y^2 + \left(\frac{U}{D} - \frac{D}{2\mu} \left(\frac{dp}{dx} + \rho gm\right) \right) y
\end{equation}
\end{document}
